\section{Lecture 5: Other Quantum Gates, Measurement, Multi-Qubit Systems}

\dfn{Single-Qubit Gates}{Quantum gates manipulate individual qubits. Key
single-qubit gates include:}
\begin{itemize}
  \item \textbf{Hadamard Gate (H):}
    \[
      \boxed{
        H = \frac{1}{\sqrt{2}} \begin{pmatrix} 1 & 1 \\ 1 & -1 \end{pmatrix}
      }
    \]

    Creates superposition: \boxed{H\zero = \ket{+}, H\one = \ket{-}}

    \vspace{0.3cm}

    Properties:

    \begin{itemize}
      \item Self-inverse: $H^2 = I$
      \item Maps computational basis to $\ket{\pm}$ basis:
        \begin{align*}
          \ket{+} &= \frac{1}{\sqrt{2}}(\zero + \one) \\
          \ket{-} &= \frac{1}{\sqrt{2}}(\zero - \one)
        \end{align*}
    \end{itemize}

    \aside{
      \[
        H \zero = \ket{+}, \quad H \one = \ket{-}, \quad H \ket{+} = \zero, \quad H \ket{-} = \one
      \]
    }

  \item \textbf{Phase Gate (S):}

    \[
      \boxed{
      S = \begin{pmatrix} 1 & 0 \\ 0 & i \end{pmatrix}
      }
    \]

    Adds a $\pi/2$ phase to $\one$, so it is also referred to as the "$\pi$/4 gate"
    due to $\theta/2$ term in the Bloch sphere equation.

    \vspace{0.3cm}

    \textbf{Properties:}

    \begin{itemize}
      \item \textbf{Unitary but not Hermitian}
      \item $S^2 = Z$
      \item Effect on $\ket{+}$ : \boxed{S\ket{+} =
        \frac{1}{\sqrt{2}}(\zero + i\one)}
    \end{itemize}

  \item \textbf{T Gate:}

    \[
      \boxed{
      T = \begin{pmatrix} 1 & 0 \\ 0 & e^{i\pi/4} \end{pmatrix}
      }
    \]

    Adds a $\pi/4$ phase to $\one$, so it is also known as the "$\pi$/8 gate".

    \vspace{0.3cm}

    \textbf{Properties:}

    \begin{itemize}
      \item $T^2 = S$
      \item $T^4 = Z$
      \item Often used in quantum error correction
    \end{itemize}

  \item \textbf{General Phase Gate } $P(\theta)$:

    \[
      P(\theta) = \begin{pmatrix} 1 & 0 \\ 0 & e^{i\theta} \end{pmatrix}
    \]

    Generalizes S and T gates: \boxed{S = P(\pi/2), T = P(\pi/4)}

\end{itemize}

\ex{Example of Applying the Hadamard Gate}{
  \begin{align*}
    H\zero &= \frac{1}{\sqrt{2}} \begin{bmatrix} 1 & 1 \\ 1 & -1
      \end{bmatrix} \begin{bmatrix} 1 \\ 0 \end{bmatrix} \\
    & \frac{1}{\sqrt{2}} \begin{bmatrix} 1 \\ 1 \end{bmatrix} = \ket{+}
  \end{align*}
}

\dfn{Important Relations}{
  \nt{
    \begin{align*}
      Z &= H X H \\
      X &= H Z H
    \end{align*}
  }

  Demonstrates duality between X and Z gates via Hadamard transformation.

  \begin{proof}
    \[
      HXH = \frac{1}{\sqrt{2}}\begin{pmatrix} 1 & 1 \\ 1 & -1 \end{pmatrix}
      \begin{pmatrix} 0 & 1 \\ 1 & 0 \end{pmatrix}
      \frac{1}{\sqrt{2}}\begin{pmatrix} 1 & 1 \\ 1 & -1 \end{pmatrix} =
      \begin{pmatrix} 1 & 0 \\ 0 & -1 \end{pmatrix} = Z
    \]
  \end{proof}
}

\subsection*{Back to Measurement}

Measurement collapses quantum states to basis states with
probabilities determined by amplitudes.

\begin{itemize}
  \item \textbf{Z-basis:} Standard computational basis ($\zero$, $\one$)

    \begin{itemize}
      \item For state $\ket{\psi} = \alpha\zero + \beta\one$:
        \begin{align*}
          P(0) &= ||\alpha||^2 \\
          P(1) &= ||\beta||^2
        \end{align*}
    \end{itemize}

  \item \textbf{X-basis:} Hadamard basis ($\ket{+}$, $\ket{-}$)

    \begin{itemize}[label={*}]
      \item Measure in Z-basis after applying H gate

      \item $P(+) = |\bra{+}\psi\rangle|^2$

      \item $P(-) = |\bra{-}\psi\rangle|^2$
    \end{itemize}

  \item \textbf{Y-basis:} Eigenstates of Y

    \begin{itemize}
      \item $\ket{+i} = \frac{1}{\sqrt{2}}(\zero + i\one)$

      \item $\ket{-i} = \frac{1}{\sqrt{2}}(\zero - i\one)$
    \end{itemize}

\end{itemize}

\dfn{Multi-Qubit Systems}{States for multiple qubits are represented as
tensor products:}

\[
  \ket{\psi} = \sum_{k=0}^{k = 2^n-1} \alpha_k \ket{k}, \quad \sum
  ||\alpha_k||^2 = 1
\]

\noindent
\textbf{Properties of tensor products:}

\begin{itemize}
  \item Not commutative: $(\ket{0} \otimes \ket{1} \neq \ket{1} \otimes
    \ket{0})$

  \item Associative: $((\ket{a} \otimes \ket{b}) \otimes \ket{c} = \ket{a}
    \otimes (\ket{b} \otimes \ket{c}))$

  \item Distributive: $((\alpha\ket{a} + \beta\ket{b}) \otimes \ket{c} =
    \alpha(\ket{a} \otimes \ket{c}) + \beta(\ket{b} \otimes \ket{c}))$
\end{itemize}

\vspace{0.3cm}

%%%%%%%%%%%%%%%%%%%%%%%%%%%

\subsection*{Exercises}
\qs{Exercise 1}{Prove that the Hadamard gate is unitary and Hermitian.}

\qs{Exercise 2}{For $\ket{\psi} = \frac{1}{\sqrt{2}}(\ket{00} + \ket{11})$,
find measurement probabilities for $\ket{00}$ and $\ket{11}$.}

\qs{Exercise 3}{Determine if $U = \frac{1}{\sqrt{2}}\begin{pmatrix} 1 & i \\
i & 1 \end{pmatrix}$ is unitary.}

\qs{Exercise 4}{If we apply H $\otimes$ H to $\ket{00}$, what state do we get?}

\sol{Exercise 1 Solution:
  To prove H is unitary and Hermitian:
  \begin{align*}
    H^\dagger &= \frac{1}{\sqrt{2}}\begin{pmatrix} 1 & 1 \\ 1 & -1
    \end{pmatrix} = H \\
    HH &= \frac{1}{2}\begin{pmatrix} 1 & 1 \\ 1 & -1
      \end{pmatrix}\begin{pmatrix} 1 & 1 \\ 1 & -1 \end{pmatrix} =
      \begin{pmatrix} 1 & 0 \\ 0 & 1 \end{pmatrix} = I
    \end{align*}
    Thus, H is both unitary ($HH^\dagger = I$) and Hermitian ($H = H^\dagger$).
}

\vspace{0.3cm}

\sol{Exercise 2 Solution:

  For $\ket{\psi} = \frac{1}{\sqrt{2}}(\ket{00} + \ket{11})$:

  \begin{align*}
    P(00) &= |\bra{00}\psi\rangle|^2 = \left|\frac{1}{\sqrt{2}}\right|^2 =
    \frac{1}{2} \\
    P(11) &= |\bra{11}\psi\rangle|^2 = \left|\frac{1}{\sqrt{2}}\right|^2 =
    \frac{1}{2}
  \end{align*}
}

\vspace{0.3cm}

\sol{Exercise 3 Solution:

  To verify unitarity, compute $UU^\dagger$:

  \begin{align*}
    U^\dagger &= \frac{1}{\sqrt{2}}\begin{pmatrix} 1 & -i \\ -i & 1
    \end{pmatrix} \\
    UU^\dagger &= \frac{1}{2}\begin{pmatrix} 1 & i \\ i & 1
      \end{pmatrix}\begin{pmatrix} 1 & -i \\ -i & 1 \end{pmatrix} =
      \begin{pmatrix} 1 & 0 \\ 0 & 1 \end{pmatrix}
  \end{align*}

  Therefore, U is unitary.
}

\vspace{0.3cm}

\sol{Exercise 4 Solution:

  \begin{align*}
    (H \otimes H)\ket{00} &= (H\ket{0}) \otimes (H\ket{0}) \\
    &= \frac{1}{\sqrt{2}}(\ket{0} + \ket{1}) \otimes
    \frac{1}{\sqrt{2}}(\ket{0} + \ket{1}) \\
    &= \frac{1}{2}(\ket{00} + \ket{01} + \ket{10} + \ket{11})
  \end{align*}

  This creates an equal superposition of all two-qubit basis states.
}
