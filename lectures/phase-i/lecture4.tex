\section{Lecture 4: Quantum Gates and Transformations}

Quantum gates manipulate qubits through unitary transformations, preserving
quantum information and enabling quantum computation. This section explores
key quantum operations, their mathematical properties, and circuit
representations.

\dfn{Qubit Superposition and Hilbert Space}{A \textbf{qubit} exists in a
  complex vector space called a \textbf{Hilbert space}. The state of a qubit
  is given by:
  \[
    |\psi\rangle = \alpha |0\rangle + \beta |1\rangle, \quad \text{where }
    \alpha, \beta \in \mathbb{C} \text{ and } |\alpha|^2 + |\beta|^2 = 1.
  \]
}

\subsection*{Measurement and Superposition Collapse}

When a qubit is measured in the computational basis $\{|0\rangle,
|1\rangle\}$, it collapses to one of the basis states with probability:

\[
  \boxed{
    P(0) = \norm{\alpha_1}^2, \quad P(1) = \norm{\alpha_2}^2.
  }
\]

The post-measurement state is:

\[
  \boxed{
    |\psi_{\text{measurement}}\rangle = \frac{|b\rangle \langle b | \psi
    \rangle}{\sqrt{P(b)}}
  }
\]

\noindent
where $b \in \{0,1\}$. This formula captures the quantum measurement
postulate and ensures proper normalization of the post-measurement state.

\vspace{0.3cm}

In the computational basis, the probability of measuring $|b\rangle$ is:

\[
  \boxed{
    P(b) = \norm{\langle b | \psi \rangle}^2
  }
\]

\nt{Probability Properties of Measurement:
  \[
    \begin{aligned}
      P(0) &= 1 - P(1) \\
      P(+) &= 1 - P(-) \\
      P(+i) &= 1 - P(-i)
    \end{aligned}
  \]
}

\subsection*{Quantum Gates and Operations}

Quantum gates are unitary matrices that transform qubits. A general qubit
transformation is given by:

\[
  |\psi_{\text{final}}\rangle = U |\psi_{\text{initial}}\rangle
\]

\noindent
where $U$ is a unitary matrix satisfying $U^\dagger U = I$. Key properties of
quantum gates include:

\begin{itemize}
  \item \textbf{Reversibility:} All quantum operations are reversible due
    to unitarity
  \item \textbf{Preservation of Norm:} The normalization condition
    $|\alpha|^2 + |\beta|^2 = 1$ is preserved
  \item \textbf{Linearity:} Gates act linearly on superposition states
\end{itemize}

\dfn{Rotation Gates}{Rotation gates rotate a qubit state around the Bloch
  sphere:
  \begin{itemize}
    \item \textbf{Rotation about X-axis:}
      \[
        \boxed{
          R_X(\omega) =
          \begin{bmatrix}
            \cos\frac{\omega}{2} & -i\sin\frac{\omega}{2} \\
            -i\sin\frac{\omega}{2} & \cos\frac{\omega}{2}
          \end{bmatrix}
        }
      \]

      Effect: Rotates state by angle $\omega$ around X-axis

    \item \textbf{Rotation about Y-axis:}
      \[
        \boxed{
          R_Y(\omega) =
          \begin{bmatrix}
            \cos\frac{\omega}{2} & -\sin\frac{\omega}{2} \\
            \sin\frac{\omega}{2} & \cos\frac{\omega}{2}
          \end{bmatrix}
        }
      \]

      Effect: Rotates state by angle $\omega$ around Y-axis

    \item \textbf{Rotation about Z-axis:}
      \[
        \boxed{
          R_Z(\omega) =
          \begin{bmatrix}
            e^{-i\omega/2} & 0 \\
            0 & e^{i\omega/2}
          \end{bmatrix}
        }
      \]

      Effect: Adds a relative phase between $|0\rangle$ and $|1\rangle$
      components
  \end{itemize}
}

\dfn{Pauli Matrices and Gates}{The \textbf{Pauli matrices} define fundamental
  quantum operations:
  \begin{itemize}
    \item \textbf{Pauli-X (NOT Gate, Bit-Flip):}
      \[
        \boxed{
          X = \begin{bmatrix} 0 & 1 \\ 1 & 0 \end{bmatrix}
        }
      \]

      Effect: $X|0\rangle = |1\rangle$, $X|1\rangle = |0\rangle$

    \item \textbf{Pauli-Y (Combination of X and Z with phase):}
      \[
        \boxed{
          Y = \begin{bmatrix} 0 & -i \\ i & 0 \end{bmatrix}
        }
      \]

      Effect: $Y|0\rangle = i|1\rangle$, $Y|1\rangle = -i|0\rangle$

    \item \textbf{Pauli-Z (Phase-Flip Gate):}
      \[
        \boxed{
          Z = \begin{bmatrix} 1 & 0 \\ 0 & -1 \end{bmatrix}
        }
      \]

      Effect: $Z|0\rangle = |0\rangle$, $Z|1\rangle = -|1\rangle$

  \end{itemize}

  \aside{Each of these matrices is both \textbf{Hermitian} ($A = A^\dagger$) and
  \textbf{unitary} ($A^\dagger A = I$).}

  Important relationships:
  \begin{itemize}
    \item $X^2 = Y^2 = Z^2 = I$
    \item $XY = iZ$, $YZ = iX$, $ZX = iY$
    \item $YX = -iZ$, $ZY = -iX$, $XZ = -iY$
\end{itemize}}

\subsection*{Circuit Notation}
Quantum circuits visually represent quantum
  operations. Each qubit is represented as a horizontal line, and gates are
  applied sequentially from left to right. Important circuit elements include:
  \begin{itemize}
    \item \textbf{Single-qubit gates:} Represented as boxes with gate symbols
    \item \textbf{Measurements:} Depicted with a meter symbol
    \item \textbf{Time flow:} Left to right in circuits (\textbf{opposite of
      matrix multiplication order})
\end{itemize}

\ex{Example: Complex Circuit Analysis}{Consider the circuit applying the
  sequence $HZH$ to $|0\rangle$:
  \[
    \begin{aligned}
      |\psi_1\rangle &= H|0\rangle = \frac{1}{\sqrt{2}}(|0\rangle +
      |1\rangle) \\
      |\psi_2\rangle &= Z|\psi_1\rangle = \frac{1}{\sqrt{2}}(|0\rangle -
      |1\rangle) \\
      |\psi_3\rangle &= H|\psi_2\rangle = |1\rangle
    \end{aligned}
  \]

  This sequence performs a NOT operation on $|0\rangle$ using only Hadamard and
  Phase-flip gates.

  \[
    \begin{quantikz}
      \lstick{$|0\rangle$} & \gate{H} & \gate{Z} & \gate{H} & \one \\
    \end{quantikz}
  \]
}

\ex{Another Circuit Example}{
  \[
    \begin{aligned}
      XY \zero & = X \begin{bmatrix} 0 & -i \\ i & 0 \end{bmatrix} \begin{bmatrix} 1 \\ 0 \end{bmatrix}\\
      & = \begin{bmatrix} 0 & 1 \\ 1 & 0 \end{bmatrix} \begin{bmatrix} 0 \\ i \end{bmatrix} \\
      & = \begin{bmatrix} i \\ 0 \end{bmatrix} = i \one
    \end{aligned}
  \]

  \[
    \begin{quantikz}
      \lstick{$\zero$} & \gate{Y} & \gate{X} & i \zero \\
    \end{quantikz}
  \]
}

%%%%%%%%%%%%%%%%%%%%%%%

\qs{Exercise 1}{Apply the sequence $SXH$ to $|0\rangle$ and calculate:
  \begin{itemize}
    \item The final state vector
    \item The probabilities of measuring $|0\rangle$ and $|1\rangle$
    \item The possible post-measurement states
\end{itemize}}

\qs{Exercise 2}{Show that the Hadamard gate is its own inverse by calculating
$H^2$.}

\qs{Exercise 3}{Calculate the effect of applying $R_Z(\pi/2)$ to the state
$|+\rangle$.}

\sol{Exercise 1 Solution:
  \begin{align*}
    H|0\rangle &= \frac{1}{\sqrt{2}}(|0\rangle + |1\rangle) \\
    XH|0\rangle &= \frac{1}{\sqrt{2}}(|1\rangle + |0\rangle) =
    \frac{1}{\sqrt{2}}(|0\rangle + |1\rangle) \\
    SXH|0\rangle &= \frac{1}{\sqrt{2}}(|0\rangle + i|1\rangle)
  \end{align*}
  Therefore:
  \begin{itemize}
    \item Final state: $|\psi\rangle = \frac{1}{\sqrt{2}}(|0\rangle +
      i|1\rangle)$
    \item Measurement probabilities: $P(0) = P(1) = \frac{1}{2}$
    \item Post-measurement states: Either $|0\rangle$ or $|1\rangle$ with
      equal probability
\end{itemize}}
