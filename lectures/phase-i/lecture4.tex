\section{Lecture 4: Quantum Gates and Transformations}

Quantum gates manipulate qubits through unitary transformations, preserving quantum information and enabling quantum computation. This section explores key quantum operations, their mathematical properties, and circuit representations.

\dfn{Qubit Superposition and Hilbert Space}{A \textbf{qubit} exists in a complex vector space called a \textbf{Hilbert space}. The state of a qubit is given by:
\[
|\psi\rangle = \alpha |0\rangle + \beta |1\rangle, \quad \text{where } \alpha, \beta \in \mathbb{C} \text{ and } |\alpha|^2 + |\beta|^2 = 1.
\]
The computational basis states are represented as:
\[
|0\rangle = \begin{pmatrix} 1 \\ 0 \end{pmatrix}, \quad |1\rangle = \begin{pmatrix} 0 \\ 1 \end{pmatrix}
\]
}

\dfn{Measurement and Superposition Collapse}{When a qubit is measured in the computational basis $\{|0\rangle, |1\rangle\}$, it collapses to one of the basis states with probability:
\[
P(0) = |\alpha|^2, \quad P(1) = |\beta|^2.
\]
The post-measurement state is:
\[
|\psi_{\text{new}}\rangle = \frac{|b\rangle \langle b | \psi \rangle}{\sqrt{P(b)}}
\]
where $b \in \{0,1\}$. This formula captures the quantum measurement postulate and ensures proper normalization of the post-measurement state.}

\dfn{Important Quantum States}{Several quantum states are particularly important in quantum computing:
\begin{itemize}
    \item \textbf{Computational Basis States:} $|0\rangle$ and $|1\rangle$
    \item \textbf{Plus/Minus States:}
    \[
    |+\rangle = \frac{1}{\sqrt{2}}(|0\rangle + |1\rangle), \quad |-\rangle = \frac{1}{\sqrt{2}}(|0\rangle - |1\rangle)
    \]
    \item \textbf{Complex Superposition States:}
    \[
    |i\rangle = \frac{1}{\sqrt{2}}(|0\rangle + i|1\rangle), \quad |-i\rangle = \frac{1}{\sqrt{2}}(|0\rangle - i|1\rangle)
    \]
\end{itemize}}

\ex{Example: Equal Superposition State}{
A qubit initially in state $|0\rangle$ is transformed into an equal superposition using the Hadamard gate:
\[
H |0\rangle = \frac{1}{\sqrt{2}} (|0\rangle + |1\rangle).
\]
Measuring this state results in either $|0\rangle$ or $|1\rangle$ with equal probability $P(0) = P(1) = \frac{1}{2}$.

Similarly, applying Hadamard to $|1\rangle$:
\[
H |1\rangle = \frac{1}{\sqrt{2}} (|0\rangle - |1\rangle) = |-\rangle
\]
}

\dfn{Quantum Gates and Operations}{Quantum gates are unitary matrices that transform qubits. A general qubit transformation is given by:
\[
|\psi_{\text{final}}\rangle = U |\psi_{\text{initial}}\rangle
\]
where $U$ is a unitary matrix satisfying $U^\dagger U = I$. Key properties of quantum gates include:
\begin{itemize}
    \item \textbf{Reversibility:} All quantum operations are reversible due to unitarity
    \item \textbf{Preservation of Norm:} The normalization condition $|\alpha|^2 + |\beta|^2 = 1$ is preserved
    \item \textbf{Linearity:} Gates act linearly on superposition states
\end{itemize}}

\dfn{Rotation Gates}{Rotation gates rotate a qubit state around the Bloch sphere:
\begin{itemize}
    \item \textbf{Rotation about X-axis:}
    \[
    R_X(\omega) =
    \begin{bmatrix}
    \cos\frac{\omega}{2} & -i\sin\frac{\omega}{2} \\
    -i\sin\frac{\omega}{2} & \cos\frac{\omega}{2}
    \end{bmatrix}
    \]
    Effect: Rotates state by angle $\omega$ around X-axis
    
    \item \textbf{Rotation about Y-axis:}
    \[
    R_Y(\omega) =
    \begin{bmatrix}
    \cos\frac{\omega}{2} & -\sin\frac{\omega}{2} \\
    \sin\frac{\omega}{2} & \cos\frac{\omega}{2}
    \end{bmatrix}
    \]
    Effect: Rotates state by angle $\omega$ around Y-axis
    
    \item \textbf{Rotation about Z-axis:}
    \[
    R_Z(\omega) =
    \begin{bmatrix}
    e^{-i\omega/2} & 0 \\
    0 & e^{i\omega/2}
    \end{bmatrix}
    \]
    Effect: Adds a relative phase between $|0\rangle$ and $|1\rangle$ components
\end{itemize}
Special cases:
\begin{itemize}
    \item $R_X(\pi) = iX$
    \item $R_Y(\pi) = iY$
    \item $R_Z(\pi) = iZ$
\end{itemize}}

\dfn{Pauli Matrices and Gates}{The \textbf{Pauli matrices} define fundamental quantum operations:
\begin{itemize}
    \item \textbf{Pauli-X (NOT Gate, Bit-Flip):}
    \[
    X = \begin{bmatrix} 0 & 1 \\ 1 & 0 \end{bmatrix}
    \]
    Effect: $X|0\rangle = |1\rangle$, $X|1\rangle = |0\rangle$
    
    \item \textbf{Pauli-Y (Combination of X and Z with phase):}
    \[
    Y = \begin{bmatrix} 0 & -i \\ i & 0 \end{bmatrix}
    \]
    Effect: $Y|0\rangle = i|1\rangle$, $Y|1\rangle = -i|0\rangle$
    
    \item \textbf{Pauli-Z (Phase-Flip Gate):}
    \[
    Z = \begin{bmatrix} 1 & 0 \\ 0 & -1 \end{bmatrix}
    \]
    Effect: $Z|0\rangle = |0\rangle$, $Z|1\rangle = -|1\rangle$
\end{itemize}
Each of these matrices is both \textbf{Hermitian} ($A = A^\dagger$) and \textbf{unitary} ($A^\dagger A = I$).

Important relationships:
\begin{itemize}
    \item $X^2 = Y^2 = Z^2 = I$
    \item $XY = iZ$, $YZ = iX$, $ZX = iY$
    \item $YX = -iZ$, $ZY = -iX$, $XZ = -iY$
\end{itemize}}

\dfn{Additional Important Gates}{
\begin{itemize}
    \item \textbf{Hadamard Gate (H):}
    \[
    H = \frac{1}{\sqrt{2}}\begin{bmatrix} 1 & 1 \\ 1 & -1 \end{bmatrix}
    \]
    Creates superposition states: $H|0\rangle = |+\rangle$, $H|1\rangle = |-\rangle$
    
    \item \textbf{Phase Gate (S):}
    \[
    S = \begin{bmatrix} 1 & 0 \\ 0 & i \end{bmatrix}
    \]
    Adds a $\pi/2$ phase to $|1\rangle$
    
    \item \textbf{T Gate:}
    \[
    T = \begin{bmatrix} 1 & 0 \\ 0 & e^{i\pi/4} \end{bmatrix}
    \]
    Adds a $\pi/4$ phase to $|1\rangle$
\end{itemize}}

\dfn{Circuit Notation}{Quantum circuits visually represent quantum operations. Each qubit is represented as a horizontal line, and gates are applied sequentially from left to right. Important circuit elements include:
\begin{itemize}
    \item \textbf{Single-qubit gates:} Represented as boxes with gate symbols
    \item \textbf{Measurements:} Depicted with a meter symbol
    \item \textbf{Time flow:} Left to right in circuits (opposite of matrix multiplication order)
    \item \textbf{Initial states:} Usually started in $|0\rangle$ unless specified otherwise
\end{itemize}}

\ex{Example: Complex Circuit Analysis}{Consider the circuit applying the sequence $HZH$ to $|0\rangle$:
\[
\begin{aligned}
|\psi_1\rangle &= H|0\rangle = \frac{1}{\sqrt{2}}(|0\rangle + |1\rangle) \\
|\psi_2\rangle &= Z|\psi_1\rangle = \frac{1}{\sqrt{2}}(|0\rangle - |1\rangle) \\
|\psi_3\rangle &= H|\psi_2\rangle = |1\rangle
\end{aligned}
\]
This sequence performs a NOT operation on $|0\rangle$ using only Hadamard and Phase-flip gates.}

\dfn{Measurement in Quantum Circuits}{Measurement collapses a quantum state to a basis state with probabilities determined by the squared magnitudes of its coefficients. For a state $|\psi\rangle = \alpha|0\rangle + \beta|1\rangle$:
\begin{itemize}
    \item Probability of measuring $|0\rangle$: $P(0) = |\alpha|^2$
    \item Probability of measuring $|1\rangle$: $P(1) = |\beta|^2$
    \item Post-measurement state is the measured basis state
    \item Multiple measurements of identically prepared states give statistical distributions
\end{itemize}}

\qs{Exercise 1}{Apply the sequence $SXH$ to $|0\rangle$ and calculate:
\begin{itemize}
    \item The final state vector
    \item The probabilities of measuring $|0\rangle$ and $|1\rangle$
    \item The possible post-measurement states
\end{itemize}}

\qs{Exercise 2}{Show that the Hadamard gate is its own inverse by calculating $H^2$.}

\qs{Exercise 3}{Calculate the effect of applying $R_Z(\pi/2)$ to the state $|+\rangle$.}

\sol{Exercise 1 Solution:
\begin{align*}
H|0\rangle &= \frac{1}{\sqrt{2}}(|0\rangle + |1\rangle) \\
XH|0\rangle &= \frac{1}{\sqrt{2}}(|1\rangle + |0\rangle) = \frac{1}{\sqrt{2}}(|0\rangle + |1\rangle) \\
SXH|0\rangle &= \frac{1}{\sqrt{2}}(|0\rangle + i|1\rangle)
\end{align*}
Therefore:
\begin{itemize}
    \item Final state: $|\psi\rangle = \frac{1}{\sqrt{2}}(|0\rangle + i|1\rangle)$
    \item Measurement probabilities: $P(0) = P(1) = \frac{1}{2}$
    \item Post-measurement states: Either $|0\rangle$ or $|1\rangle$ with equal probability
\end{itemize}}
