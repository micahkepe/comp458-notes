\section{Lecture 7:  More Multi-Qubit Gates, Reversibility Property,
No-Cloning Theorem}\label{sec:lecture7}

\subsection*{Review Questions}

\qs{Number of Measurement Bases}{
  How many different bases can we measure a $n$-qubit system in?
}

\sol{\boxed{\textbf{Infinite}}

  \[
    P(b) = \norm{\langle b | \psi \rangle}^2
  \]

  In quantum mechanics, a measurement basis for an $n$-qubit system is a set of
  orthonormal basis states in a $2^n$-dimensional Hilbert space. The most
  common measurement basis is the \textbf{computational basis}, given by
  $\{|0\rangle, |1\rangle\}^{\otimes n}$. However, we can measure in
  \textbf{any} orthonormal basis.

  The space of all possible measurement bases corresponds to the space of all
  possible orthonormal bases, which is parameterized by the unitary group
  $U(2^n)$. The set of all $2^n$-dimensional orthonormal bases is described
  by the unitary group $U(2^n)$, modulo the global phase $U(1)$. Since this
  space is continuous and has infinitely many parameters, there exist an
  \textbf{infinite} number of measurement bases.
}

\qs{Output of Quantum Circuit}{

  What is the output quantum state of the following quantum circuit:

  \[
    \begin{quantikz}
      q_2 \quad \lstick{$\zero$} & \gate{X} & \gate{X} & \meter{} \\
      q_1 \quad \lstick{$\zero$} & \gate{S} & \gate{X} & \meter{} \\
      q_0 \quad \lstick{$\zero$} & \gate{X} & & \meter{} \\
    \end{quantikz}
  \]
}

\sol{
  \boxed{011}

  \begin{itemize}
    \item For $q_2$, the two $X$ gates essentially cancel each other out as
      the gate is Hermitian ($XX = I$).
    \item For $q_1$, the $S$ gate does not affect the starting state $\zero$,
      and then the $X$ gate flips the signal to $\one$.
    \item For $q_0$, the $X$ gate simply flips the signal from $\zero$ to
      $\one$.
  \end{itemize}

  Remembering that we read the diagram top-down as the most significant bit
  to the least significant bit, respectively, the output is 001.
}

\subsection*{More Multi-Qubit Gates}

\subsubsection*{More 2-Qubit Gates}

\index{quantum gates!multi-qubit gates!CNOT with swapped target and control}
\dfn{$CX(q_0 \rightarrow q_1)$ Gate}{

  The control and target qubits of a CNOT gate can be swapped using Hadamard
  gates:

  \[
    (H \otimes H) \cdot CNOT_{\text{control,target}} \cdot (H \otimes H) =
    CNOT_{\text{target,control}}
  \]

  This transformation can be visualized in a circuit diagram:

  \begin{align*}
    \begin{quantikz}
      \lstick{$\ket{c}$} & \gate{H} & \ctrl{1} & \gate{H} & \qw \\
      \lstick{$\ket{t}$} & \gate{H} & \gate{X} & \gate{H} & \qw \\
    \end{quantikz}
  \end{align*}

  The $CX(q_0 \rightarrow q_1)$ gate, also known as $CNOT_{target,control}$,
  is a variant of the CNOT gate where the control and target qubits are
  swapped. Its matrix representation is:

  \[
    CNOT_{target,control} = \begin{pmatrix}
      1 & 0 & 0 & 0 \\
      0 & 0 & 0 & 1 \\
      0 & 0 & 1 & 0 \\
      0 & 1 & 0 & 0
    \end{pmatrix}
  \]

  This gate flips the target qubit ($q_1$) if the control qubit ($q_0$) is in
  state $\ket{1}$.
}

\subsubsection*{Derivation of the $CX(q_0 \rightarrow q_1)$ Gate}

The control and target qubits of a CNOT gate can be swapped using Hadamard
($H$) gates applied to both qubits. Mathematically, this operation is
represented as:

\[
  (H \otimes H) \cdot \text{CNOT}_{\text{control,target}} \cdot (H \otimes H)
  = \text{CNOT}_{\text{target,control}}
\]


The transformed CNOT gate is:

\[
  (H \otimes H) \cdot \text{CNOT} \cdot (H \otimes H).
\]

\vspace{0.3cm}

\textbf{Step 1: Compute $\text{CNOT} \cdot (H \otimes H)$}

First, multiply the CNOT matrix with $H \otimes H$:

\[
  \text{CNOT} \cdot (H \otimes H) = \begin{pmatrix}
    1 & 0 & 0 & 0 \\
    0 & 1 & 0 & 0 \\
    0 & 0 & 0 & 1 \\
    0 & 0 & 1 & 0
    \end{pmatrix} \cdot \frac{1}{2} \begin{pmatrix}
    1 & 1 & 1 & 1 \\
    1 & -1 & 1 & -1 \\
    1 & 1 & -1 & -1 \\
    1 & -1 & -1 & 1
  \end{pmatrix}
\]

This results in:

\[
  \text{CNOT} \cdot (H \otimes H) = \frac{1}{2} \begin{pmatrix}
    1 & 1 & 1 & 1 \\
    1 & -1 & 1 & -1 \\
    1 & -1 & -1 & 1 \\
    1 & 1 & -1 & -1
  \end{pmatrix}
\]

Notice that the third and fourth rows of the matrix switch places due to the
CNOT gate's effect.

\vspace{0.3cm}

\textbf{Step 2: Multiply $(H \otimes H)$ to the Above Result}

Now, multiply $H \otimes H$ from the left:

\[
  (H \otimes H) \cdot \frac{1}{2} \begin{pmatrix}
    1 & 1 & 1 & 1 \\
    1 & -1 & 1 & -1 \\
    1 & -1 & -1 & 1 \\
    1 & 1 & -1 & -1
  \end{pmatrix}
\]

Expanding this:

\[
  \frac{1}{2} \begin{pmatrix}
    1 & 1 & 1 & 1 \\
    1 & -1 & 1 & -1 \\
    1 & 1 & -1 & -1 \\
    1 & -1 & -1 & 1
    \end{pmatrix} \cdot \frac{1}{2} \begin{pmatrix}
    1 & 1 & 1 & 1 \\
    1 & -1 & 1 & -1 \\
    1 & -1 & -1 & 1 \\
    1 & 1 & -1 & -1
  \end{pmatrix}
\]

After computation, the resulting matrix is:

\[
  \text{CNOT}_{\text{target,control}} = \begin{pmatrix}
    1 & 0 & 0 & 0 \\
    0 & 0 & 0 & 1 \\
    0 & 0 & 1 & 0 \\
    0 & 1 & 0 & 0
  \end{pmatrix}
\]

This matrix corresponds to the CNOT gate with the control and target qubits
swapped.


\index{quantum gates!multi-qubit gates!SWAP gate}
\subsection*{SWAP Gate}

\dfn{SWAP Gate}{
  The SWAP gate exchanges the states of two qubits. Its matrix representation
  is:

  \[
    SWAP = \begin{pmatrix}
      1 & 0 & 0 & 0 \\
      0 & 0 & 1 & 0 \\
      0 & 1 & 0 & 0 \\
      0 & 0 & 0 & 1
    \end{pmatrix}
  \]

  The action of SWAP on basis states is given by:

  \[
    SWAP\ket{ab} = \ket{ba}, \quad a,b \in \{0,1\}
  \]
}


\paragraph{Effects of SWAP Gate}\label{par:Effects of SWAP Gate}
The SWAP gate interchanges the states of two qubits. For any 2-qubit
computational basis state, its action is:

\nt{
  \[
    \begin{aligned}
      SWAP\,\ket{00} &= \ket{00}, \\
      SWAP\,\ket{01} &= \ket{10}, \\
      SWAP\,\ket{10} &= \ket{01}, \\
      SWAP\,\ket{11} &= \ket{11}.
    \end{aligned}
  \]
}

Thus, for any superposition of 2-qubit states, the SWAP gate exchanges the
amplitudes corresponding to each qubit's position.


\index{quantum gates!$n$-qubit gates}
\subsection*{$n$-Qubit Gates}

Before we just looked at the 2-qubit version of the Controlled X gate, but
it extends to $n$-qubits.

\index{quantum gates!multi-qubit gates!Toffoli gate}
\subsection*{Toffoli Gate}

\dfn{Toffoli Gate}{(CCX) is a three-qubit gate with two control qubits and
  one target qubit. Its matrix representation is:

  \[
    \textsc{Toffoli} = \begin{pmatrix}
      1 & 0 & 0 & 0 & 0 & 0 & 0 & 0 \\
      0 & 1 & 0 & 0 & 0 & 0 & 0 & 0 \\
      0 & 0 & 1 & 0 & 0 & 0 & 0 & 0 \\
      0 & 0 & 0 & 1 & 0 & 0 & 0 & 0 \\
      0 & 0 & 0 & 0 & 1 & 0 & 0 & 0 \\
      0 & 0 & 0 & 0 & 0 & 1 & 0 & 0 \\
      0 & 0 & 0 & 0 & 0 & 0 & 0 & 1 \\
      0 & 0 & 0 & 0 & 0 & 0 & 1 & 0
    \end{pmatrix}
  \]

  The Toffoli gate is Hermitian and only flips the target qubit if both control
  qubits are in state $\ket{1}$.
}

\paragraph{Effects of Toffoli Gate}\label{par:Effects of Toffoli Gate}

The Toffoli (CCX) gate acts on a 3-qubit system, where the first two qubits
serve as control qubits and the third is the target. Its effect on the
computational basis states is:

\nt{
  \[
    \begin{aligned}
      \text{Toffoli}\,\ket{000} &= \ket{000}, \\
      \text{Toffoli}\,\ket{001} &= \ket{001}, \\
      \text{Toffoli}\,\ket{010} &= \ket{010}, \\
      \text{Toffoli}\,\ket{011} &= \ket{011}, \\
      \text{Toffoli}\,\ket{100} &= \ket{100}, \\
      \text{Toffoli}\,\ket{101} &= \ket{101}, \\
      \text{Toffoli}\,\ket{110} &= \ket{111}, \\
      \text{Toffoli}\,\ket{111} &= \ket{110}.
    \end{aligned}
  \]
}

In essence, the target qubit is flipped only when both control qubits are in
the \(\ket{1}\) state; otherwise, the state remains unchanged.

As you would expect, multi-controlled $X$ gates are Hermitian.

\ex{Quantum Circuit Showing Hermitian Property}{
  A quantum circuit composed of Hermitian gates (e.g., $X$, $Z$, $H$) will
  cancel out when the circuit is reversed, resulting in the identity operation.

  \begin{center}
    \begin{quantikz}
      \lstick{\ket{\psi}} & \gate{H} & \gate{Z} & \gate{X} & \gate{X} &
      \gate{Z} & \gate{H} & \qw & \rstick{\ket{\psi}}
    \end{quantikz}
  \end{center}

  In this example, the quantum circuit consists of a sequence of Hermitian
  gates: the Hadamard ($H$), Pauli-Z ($Z$), and Pauli-X ($X$) gates. These
  gates satisfy the property:

  \[
    H = H^\dagger, \quad X = X^\dagger, \quad Z = Z^\dagger
  \]

  Since these gates are their own inverses (\textit{i.e.,} $H H = I$, $X X =
  I$, and $Z Z = I$), if we apply the same sequence of gates in reverse
  order, they cancel out, leaving the identity operation.

  The circuit above first applies $H$, then $Z$, then $X$ twice (which
  cancels itself out), then $Z$ again, and finally $H$ again. This results in:

  \[
    H Z X X Z H = I
  \]

  Hence, the overall operation on the qubit is the identity transformation,
  meaning the final state remains the same as the initial state $\ket{\psi}$.
}

\vspace{0.3cm}

\noindent\rule{\linewidth}{0.2pt}

\vspace{0.3cm}

\index{Reversibility Property of Quantum Computing}
\dfn{Reversibility Property of Quantum Computing}{
  Quantum operations are inherently reversible due to the \textbf{unitary}
  nature of quantum gates. This means that any quantum circuit can be
  reversed by applying the inverse of each gate in the reverse order.
  Mathematically, if a quantum circuit is represented by a unitary matrix
  $U$, its reverse is represented by $U^{\dagger}$, and since quantum gates
  are unitary, they satisfy the property:

  \[
    U^{\dagger} U = U U^{\dagger} = I
  \]

  This reversibility is a fundamental difference between quantum and classical
  computing and is directly tied to the \textbf{no information loss}
  principle in quantum mechanics.
}

\paragraph{Why Reversibility is Important:}
In classical computing, operations such as the AND gate lose information.
For example, given the output of an AND gate, we cannot uniquely determine
the original input:

\[
  (0,0) \mapsto 0, \quad (0,1) \mapsto 0, \quad (1,0) \mapsto 0, \quad
  (1,1) \mapsto 1
\]

Since multiple inputs can produce the same output, information is lost,
making the operation \textbf{irreversible}.

\paragraph{Classical AND Gate (Irreversible)}
\begin{center}
  \begin{tabular}{cc}
    % Circuit diagram
    \begin{circuitikz} \draw
      % Input nodes and AND gate
      (0,1) node[left] {$A$} -- ++(1,0)
      (0,0) node[left] {$B$} -- ++(1,0)
      (2,0.5) node[and port] (and) {}
      % Connect everything
      (1,1) -- (and.in 1)
      (1,0) -- (and.in 2)
      (and.out) -- ++(1,0) node[right] {$A \land B$};
    \end{circuitikz}
    &
    % Truth table
    \begin{tabular}{|c|c|c|}
      \hline
      $A$ & $B$ & $A \land B$ \\
      \hline
      0 & 0 & 0 \\
      0 & 1 & 0 \\
      1 & 0 & 0 \\
      1 & 1 & 1 \\
      \hline
    \end{tabular}
  \end{tabular}
\end{center}

\noindent This classical AND gate demonstrates information loss
(irreversibility) because multiple distinct input states map to the same
output state. As shown in the truth table, three different input
combinations (0,0), (0,1), and (1,0) all produce the same output 0. Given
only the output 0, it is impossible to determine which of these three input
states generated it—this loss of information about the system's initial
state makes the operation irreversible.

\paragraph{Quantum Reversibility:}
In contrast, \textbf{quantum gates are always unitary}, meaning they
preserve the total amount of information. Given the final state of a
quantum system, we can always determine its previous state by applying the
inverse transformation. This is why quantum circuits must be composed of
reversible operations.

\ex{Example of a Reversible Quantum Circuit:}{
  \begin{center}
    \begin{quantikz}
      \lstick{\ket{\psi}} & \gate{H} & \gate{X} & \gate{X} & \gate{H} & \qw &
      \rstick{\ket{\psi}}
    \end{quantikz}
  \end{center}

  In this example, the quantum circuit consists of a sequence of unitary
  gates: the Hadamard ($H$) and the Pauli-X ($X$) gates. These gates satisfy:

  \[
    H = H^\dagger, \quad X = X^\dagger
  \]

  Since these gates are their own inverses (\textit{i.e.,} $H H = I$ and $X X
  = I$), if we apply the same sequence of gates in reverse order, they cancel
  out, leaving the identity operation:

  \[
    H X X H = I
  \]
}

\paragraph{Key Properties of Reversible Quantum Gates:}
\begin{enumerate}
  \item Every quantum gate \( U \) has an inverse \( U^\dagger \), ensuring
    that no information is lost.
  \item The composition of unitary gates remains unitary, preserving
    reversibility.
  \item Classical Toffoli and Fredkin\footnote{More on Fredkin gates: \url{https://en.wikipedia.org/wiki/Fredkin_gate}}
    are reversible and can be used to construct reversible classical circuits,
    which is why they are also fundamental in quantum computing.
  \item Measurement is \textbf{not} reversible, as it collapses the quantum
    state and introduces information loss.
\end{enumerate}

The reversibility of quantum computing is crucial for error correction,
fault-tolerant quantum computation, and simulating physical systems where
information is conserved.


\ex{Example of Reversing a Quantum Circuit}{
  Consider a circuit composed of gates $A$, $B$, and $C$:

  \[
    |\psi_{\text{final}}\rangle = CBA|\psi_{\text{initial}}\rangle.
  \]

  The reverse circuit applies $C^{\dagger}$, $B^{\dagger}$, and $A^{\dagger}$
  in reverse order:

  \[
    |\psi_{\text{reversed}}\rangle =
    A^{\dagger}B^{\dagger}C^{\dagger}|\psi_{\text{final}}\rangle =
    A^{\dagger}B^{\dagger}C^{\dagger}CBA|\psi_{\text{initial}}\rangle =
    |\psi_{\text{initial}}\rangle.
  \]
}


\index{Quantum No-Cloning Theorem}
\dfn{Quantum No-Cloning Theorem}{
  The no-cloning theorem states that it is impossible to create an identical
  copy of an arbitrary unknown quantum state. This result follows directly
  from the linearity of quantum mechanics.
}

\ex{Simple 2-Qubit Example}{

  Consider two qubits in states $\ket{\psi}$ and $\ket{0}$. The no-cloning
  theorem implies that there is no unitary operation $U$ such that:

  \[
    U(\ket{\psi} \otimes \ket{0}) = \ket{\psi} \otimes \ket{\psi}.
  \]
}

\index{Quantum No-Cloning Theorem!proof@\textit{proof}}
\begin{proof}
  \textbf{Proof of the Quantum No-Cloning Theorem}

  Assume a unitary operator $U$ exists that can clone an arbitrary quantum
  state $\ket{\psi}$. Then, for two different states $\ket{\psi}$ and
  $\ket{\phi}$, we would have:

  \[
    U(\ket{\psi} \otimes \ket{0}) = \ket{\psi} \otimes \ket{\psi},
  \]

  \[
    U(\ket{\phi} \otimes \ket{0}) = \ket{\phi} \otimes \ket{\phi}.
  \]

  Now, consider the superposition state $\ket{\xi} = a\ket{\psi} +
  b\ket{\phi}$. Applying $U$ to $\ket{\xi} \otimes \ket{0}$ should produce:

  \[
    U(\ket{\xi} \otimes \ket{0}) = aU(\ket{\psi} \otimes \ket{0}) +
    bU(\ket{\phi} \otimes \ket{0}) = a(\ket{\psi} \otimes \ket{\psi}) +
    b(\ket{\phi} \otimes \ket{\phi}).
  \]

  However, if $U$ could clone $\ket{\xi}$, the result should be:

  \[
    U(\ket{\xi} \otimes \ket{0}) = \ket{\xi} \otimes \ket{\xi} = (a\ket{\psi}
    + b\ket{\phi}) \otimes (a\ket{\psi} + b\ket{\phi}).
  \]

  Expanding this, we get:

  \[
    \ket{\xi} \otimes \ket{\xi} = a^2(\ket{\psi} \otimes \ket{\psi}) +
    ab(\ket{\psi} \otimes \ket{\phi}) + ab(\ket{\phi} \otimes \ket{\psi}) +
    b^2(\ket{\phi} \otimes \ket{\phi}).
  \]

  Comparing the two expressions, we see that the terms $\ket{\psi} \otimes
  \ket{\phi}$ and $\ket{\phi} \otimes \ket{\psi}$ appear in the expanded
  $\ket{\xi} \otimes \ket{\xi}$, but they do not appear in $a(\ket{\psi}
  \otimes \ket{\psi}) + b(\ket{\phi} \otimes \ket{\phi})$. This inconsistency
  demonstrates that a unitary operator $U$ cannot clone an arbitrary quantum
  state, proving the no-cloning theorem.
\end{proof}


%%%%%%%%%%%%%%%%%%%%%%%%%%%%%%%%%%%%%%%%%%%%%%
% End of Lecture 7
%%%%%%%%%%%%%%%%%%%%%%%%%%%%%%%%%%%%%%%%%%%%%%
