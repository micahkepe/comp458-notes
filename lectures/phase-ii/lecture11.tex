\section{Lecture 11: Quantum Circuit Optimizations}\label{sec:lecture11}

The biggest sources of inefficiencies from the Grover's Search circuit that
we saw at the end of the last lecture were:

\begin{enumerate}
  \item The way in which we implemented the multi-way OR gates.

  \item The extensive use of ancilla qubits.
\end{enumerate}

\vspace{0.3cm}

\noindent
We will address both of these inefficiencies to get a shallower and simpler
circuit.

\index{quantum gates!OR gate!optimizations@\textit{optimizations}}
\subsection*{More Efficient OR Gate Implementation}

The original OR gate implementations (e.g., \texttt{twoway\_OR},
\texttt{threeway\_OR}) in Lecture 10 used multiple Toffoli and CNOT gates,
leading to high depth and gate redundancy. For example, the three-way OR ($d
= a \lor b \lor c$) had a depth of 6. We optimize this using a single
multi-controlled NOT gate:

\begin{minted}[linenos,highlightlines={6-8}]{python}
def OR(circuit, qubits, nots):
  for index, is_not in enumerate(nots):
    if not is_not:
      circuit.append(cirq.X(qubits[index]))
  circuit.append(cirq.X(qubits[-1]))
  circuit.append(cirq.X(qubits[-1]).controlled_by(*qubits[:-1]))
  for index, is_not in enumerate(nots):
    if not is_not:
      circuit.append(cirq.X(qubits[index]))
  return circuit
\end{minted}

\textbf{Explanation:}
\begin{itemize}
  \item \textit{Lines 6-8}: Inverts the output qubit, then applies a
    multi-controlled $X$ (e.g., Toffoli for 2 inputs, CCCNOT for 3) to flip
    it back only if all inputs are 0 (after negation if needed). This
    computes $a \lor b \lor c$ efficiently.

  \item \textit{Negations}: $X$ gates preprocess inputs based on
    \texttt{nots} (0 for normal, 1 for negated), e.g., $\neg a \lor \neg b$
    uses $X$ on $a$ and $b$.
\end{itemize}

\textbf{Circuit Diagram (e.g., $\neg a \lor \neg b$):}
\[
 \begin{quantikz}
  \lstick{$\ket{a}$} & \gate{X} & \ctrl{1} & \gate{X} & \qw \\
  \lstick{$\ket{b}$} & \gate{X} & \ctrl{1} & \gate{X} & \qw \\
  \lstick{$\ket{c}$} & \qw & \targ{} & \qw & \qw
\end{quantikz}
\]

\textbf{Truth Table:}
\[
  \begin{array}{cc|c}
    a & b & c = \neg a \lor \neg b \\
    \hline
    0 & 0 & 1 \\
    0 & 1 & 1 \\
    1 & 0 & 1 \\
    1 & 1 & 0 \\
  \end{array}
\]

\textbf{Advantages:}
\begin{itemize}
  \item Reduces depth to 3 (two $X$ layers + one multi-controlled $X$) vs. 5
    in \texttt{twoway\_OR}.

  \item Eliminates redundant CNOTs, e.g., in \texttt{threeway\_OR}, depth
    drops from 6 to 4.

  \item Scales efficiently for $n$-way OR with one $n$-controlled $X$ gate.

\end{itemize}

Applying this to the Alexandria ship clauses (e.g., $\neg A \lor \neg C$,
$\neg A \lor J \lor P$):

\begin{minted}{python}
  OR(clauses_circuit, [qubits[0], qubits[1], qubits[4]], [1, 1])  # NOT A OR NOT C
  OR(clauses_circuit, [qubits[0], qubits[2], qubits[3], qubits[5]], [1, 0, 0])  # NOT A OR J OR P
\end{minted}

This reduces gate count and depth per clause, improving the oracle’s efficiency.

\subsection*{Reducing Ancilla Qubits}

\index{algebraic normal form}
\subsubsection*{Converting from CNF to ANF}

The Lecture 10 CNF implementation used 6 ancilla qubits for the Alexandria
ship problem’s clauses, scaling as $O(\text{\# of clauses})$. Converting to
Algebraic Normal Form (ANF) reduces this to $O(1)$ ancilla—typically one—by
expressing the SAT problem as a single XOR polynomial.

\textbf{CNF (Lecture 10):}

\[
  (\neg A \lor \neg C) \land (\neg A \lor J \lor P) \land (\neg C \lor \neg
  J) \land (\neg C \lor \neg P) \land J \land (J \lor \neg P)
\]

\textbf{ANF Conversion (using SymPy):}
\begin{minted}{python}
from sympy import symbols, ANFform
P, J, C, A = symbols('P, J, C, A')
phrase = (~A | ~C) & (~A | J | P) & (~C | ~J) & (~C | ~P) & J & (J | ~P)
truth_table = [int(bool(phrase.subs({P: int(seq[0]), J: int(seq[1]), C: int(seq[2]), A: int(seq[3])})))
               for seq in itertools.product("01", repeat=4)]
anf = ANFform([P, J, C, A], truth_table)  # Output: J XOR C J
\end{minted}

\textbf{Resulting ANF:} $f = J \oplus C J$ (simplified form from truth table).

\textbf{Truth Table Verification:}

\[
  \begin{array}{cccc|c}
    P & J & C & A & f \\
    \hline
    0 & 0 & 0 & 0 & 0 \\
    0 & 0 & 0 & 1 & 0 \\
    0 & 0 & 1 & 0 & 0 \\
    0 & 0 & 1 & 1 & 0 \\
    0 & 1 & 0 & 0 & 1 \\
    0 & 1 & 0 & 1 & 1 \\
    0 & 1 & 1 & 0 & 0 \\
    0 & 1 & 1 & 1 & 0 \\
    1 & 0 & 0 & 0 & 0 \\
    1 & 0 & 0 & 1 & 0 \\
    1 & 0 & 1 & 0 & 0 \\
    1 & 0 & 1 & 1 & 0 \\
    1 & 1 & 0 & 0 & 1 \\
    1 & 1 & 0 & 1 & 1 \\
    1 & 1 & 1 & 0 & 0 \\
    1 & 1 & 1 & 1 & 0 \\
  \end{array}
\]

Matches Lecture 10 solutions (e.g., 4=0100, 5=0101, 12=1100, 13=1101).

\vspace{0.3cm}

\textbf{Oracle Implementation:}
\begin{minted}{python}
oracle_circuit = cirq.Circuit()
qubits = cirq.LineQubit.range(5)  # A=0, C=1, J=2, P=3, out=4
oracle_circuit.append(cirq.CNOT(qubits[2], qubits[4]))  # J
oracle_circuit.append(cirq.CCNOT(qubits[1], qubits[2], qubits[4]))  # C J
\end{minted}

\vspace{0.3cm}

\noindent
\textbf{Circuit Diagram:}
\[
\begin{quantikz}
  \lstick{$\ket{A}$} & \qw & \qw & \qw \\
  \lstick{$\ket{C}$} & \qw & \ctrl{3} & \qw \\
  \lstick{$\ket{J}$} & \ctrl{2} & \ctrl{1} & \qw \\
  \lstick{$\ket{P}$} & \qw & \qw & \qw \\
  \lstick{$\ket{out}$} & \targ{} & \targ{} & \qw
\end{quantikz}
\]

\vspace{0.3cm}

\noindent
\textbf{Advantages:}
\begin{itemize}
  \item \textit{Ancilla}: Only 1 ancilla vs. 6, reducing qubit overhead.

  \item \textit{Depth}: 2 gates (CNOT, Toffoli) vs. multiple OR gates plus a
    6-controlled $X$.

  \item \textit{Generality}: ANF conversion applies to any CNF, often
    simplifying to fewer terms.
\end{itemize}

\vspace{0.3cm}

\noindent
\textbf{Full Circuit Results (8192 runs):}
\begin{verbatim}
Counter({5: 2075, 13: 2060, 4: 2047, 12: 2010})
\end{verbatim}

Matches Lecture 10, confirming correctness.

\vspace{0.3cm}

\noindent
\textbf{Single Winner Case (A AND NOT C AND J AND P):}
\begin{minted}{python}
phrase = A & ~C & J & P
anf = ANFform([P, J, C, A], truth_table)  # P J A XOR P J C A
oracle_circuit.append(cirq.X(qubits[4]).controlled_by(qubits[0], qubits[2], qubits[3]))
oracle_circuit.append(cirq.X(qubits[4]).controlled_by(qubits[0], qubits[1], qubits[2], qubits[3]))
\end{minted}

Results (3 iterations): High probability for 13=1101 ($P=1, J=1, C=0, A=1$).

\vspace{0.3cm}

These optimizations reduce gate redundancy and ancilla usage, enhancing
Grover’s efficiency for SAT problems.

\vspace{0.3cm}

Looking at the Cirq circuit output, we can see how much simpler the circuit
looks:

\begin{verbatim}
0: ---H---------------H---X-------@---X---H-------M-----------
                                  |               |
1: ---H-----------@---H---X-------@---X---H-------M-----------
                  |               |               |
2: ---H-------@---@---H---X-------@---X---H-------M-----------
              |   |               |               |
3: ---H---------------H---X---H---X---H---X---H---M('PJCA')---
              |   |
4: ---X---H---X---X-------------------------------------------
\end{verbatim}

