\section{Lecture 22: Quantum Annealers, Trapped Ion Quantum Computers}
\label{sec:lecture22}

\index{quantum annealers}
\subsection*{Quantum Annealers}

Quantum annealing is a metaheuristic for finding the global minimum of a
given objective function by exploiting quantum mechanical effects such as
quantum tunneling and quantum entanglement.

\subsubsection*{Basic Principles}
Quantum annealing is a physical implementation of quantum adiabatic
computation, which relies on the adiabatic theorem of quantum mechanics. The
system is initialized in a quantum superposition state (typically the ground
state of a simple Hamiltonian), and then slowly evolved to a final
Hamiltonian whose ground state encodes the solution to the problem.

\dfn{Quantum Annealing}{
  Quantum annealing is an optimization process that transforms a quantum
  system from an initial Hamiltonian $H_i$ with a known ground state to a
  final problem Hamiltonian $H_f$ whose ground state represents the solution
  to the optimization problem. The time-dependent Hamiltonian follows:

  \begin{equation*}
    H(t) = A(t)H_i + B(t)H_f
  \end{equation*}

  where $A(t)$ decreases from $A(0) = 1$ to $A(T) = 0$ and $B(t)$ increases
  from $B(0) = 0$ to $B(T) = 1$ over annealing time $T$.
}

\subsubsection*{D-Wave Systems}

The most prominent implementation of quantum annealing hardware is by D-Wave
Systems, which has progressed through multiple generations:

\begin{itemize}
  \item D-Wave One (2011): 128 qubits
  \item D-Wave Two (2013): 512 qubits
  \item D-Wave 2X (2015): 1,000+ qubits
  \item D-Wave Advantage (2020): 5,000+ qubits
  \item D-Wave Advantage2 (2023): Over 7,000 qubits
\end{itemize}

D-Wave's quantum annealers are particularly suited for solving quadratic
unconstrained binary optimization (QUBO) problems, which can be mapped to the
Ising model:

\begin{equation}
  H = \sum_{i} h_i \sigma_i^z + \sum_{i<j} J_{ij} \sigma_i^z \sigma_j^z
\end{equation}

where $\sigma_i^z$ are Pauli-Z operators, $h_i$ are local fields, and
$J_{ij}$ are coupling strengths between qubits.

\subsubsection*{Hardware Architecture}

D-Wave's quantum processing units (QPUs) consist of superconducting flux
qubits arranged in a specific topology:

\begin{itemize}
  \item Early systems: Chimera graph topology
  \item Newer systems: Pegasus and Zephyr topologies with higher connectivity
\end{itemize}

The limited connectivity between qubits requires problem embedding, which
maps logical problems onto the physical qubit topology, often using multiple
physical qubits to represent a single logical qubit.

\subsubsection*{Applications and Limitations}

Quantum annealers have been applied to various optimization problems:
\begin{itemize}
  \item Portfolio optimization
  \item Traffic flow optimization
  \item Machine learning (training of Boltzmann machines)
  \item Drug discovery
  \item Material design
\end{itemize}

\vspace{0.3cm}

\noindent
However, several limitations exist:

\begin{itemize}
  \item Limited connectivity requiring complex embeddings
  \item Quantum effects may be suppressed by noise and decoherence
  \item Difficulty in verifying "quantumness" and quantum advantage
  \item Problem size limitations due to available qubits and connectivity
\end{itemize}

\subsubsection*{Quantum Advantage Debate}
The question of whether D-Wave's systems provide quantum advantage remains
contentious. While some studies have shown modest speedups for specific
problems, definitive evidence of exponential quantum advantage has not been
demonstrated. Recent research suggests that quantum annealers may provide
advantages for problems with specific structure, particularly those with
tall, narrow energy barriers where quantum tunneling offers benefits over
classical simulated annealing.

\subsubsection*{Recent Developments}
Recent advances in quantum annealing include:
\begin{itemize}
  \item Reverse annealing protocols that start from candidate solutions
  \item Pause-and-quench annealing schedules
  \item Improved error correction techniques
  \item Hybrid quantum-classical algorithms that combine quantum annealing
    with classical optimization
\end{itemize}

\aside{
  While quantum annealers like D-Wave's systems represent an alternative
  approach to gate-based quantum computing, they are specialized machines
  designed for optimization problems rather than general-purpose quantum
  computers. The debate about their quantum advantage continues, with
  researchers exploring specific problem classes where quantum annealers
  might outperform classical methods.
}

\index{trapped ion quantum computers}
\subsection*{Trapped Ion Quantum Computers}

Trapped ion quantum computers represent one of the most promising and mature
platforms for quantum information processing, offering exceptional qubit
coherence times and high-fidelity gate operations.

\subsubsection*{Fundamental Principles}

In trapped ion quantum computing, individual ions (typically alkaline earth
metal ions like Ca$^+$, Sr$^+$, or rare earth ions like Yb$^+$) are confined
using electromagnetic fields in ultra-high vacuum environments.

\dfn{Ion Trap}{
  An ion trap is a device that uses electric or magnetic fields to capture
  charged particles in a confined region of space. The most common type used
  in quantum computing is the linear Paul trap, which uses oscillating
  electric fields to create a pseudopotential that confines ions in a linear
  chain.
}

\vspace{0.3cm}

\noindent
Qubits are encoded in the internal electronic states of each ion, typically
using:

\begin{itemize}
  \item Hyperfine ground states (e.g., $^2$S$_{1/2}$ states in $^{171}$Yb$^+$)
  \item Zeeman sublevels of ground states
  \item Optical transitions between ground and metastable excited states
\end{itemize}

\subsubsection*{Quantum Operations}

\begin{itemize}
  \item \textbf{Initialization:} Optical pumping techniques prepare ions in
    well-defined initial states with high fidelity.

  \item \textbf{Single-qubit gates:} Implemented using precisely controlled
    laser pulses or microwave radiation that drive transitions between qubit
    states. These operations can achieve fidelities exceeding 99.9\%.

  \item \textbf{Two-qubit gates:} Exploit the shared motional modes of the
    ion chain. The Coulomb interaction between ions creates collective
    vibrational modes that can mediate entangling operations. Common
    implementations include:
    \begin{itemize}
      \item M\o lmer–S\o rensen gates
      \item Cirac-Zoller gates
      \item Geometric phase gates
    \end{itemize}

  \item \textbf{Readout:} State-dependent fluorescence detection, where one
    qubit state fluoresces when illuminated by a laser while the other
    remains dark, allowing for high-fidelity measurement.
\end{itemize}

\begin{example}{M\o lmer–S\o rensen Gate}
  The M\o lmer–S\o rensen gate uses bichromatic laser fields to drive
  transitions that entangle pairs of ions without populating the motional
  modes, making it robust against heating. The interaction Hamiltonian can be
  written as:
  \begin{equation}
    H_{MS} = \sum_{i,j} J_{ij} \sigma_i^+ \sigma_j^+ + \sigma_i^- \sigma_j^-
  \end{equation}
  where $\sigma^+$ and $\sigma^-$ are the raising and lowering operators, and
  $J_{ij}$ represents the coupling strength.
\end{example}

\subsubsection*{Key Advantages}

Trapped ion quantum computers offer several notable advantages:

\begin{itemize}
  \item \textbf{Long coherence times:} Ions are naturally isolated from their
    environment, with coherence times ranging from seconds to minutes, orders
    of magnitude longer than most solid-state qubits.

  \item \textbf{Identical qubits:} As fundamental atomic systems, each ion of
    a given species is identical, eliminating the variability found in
    manufactured solid-state qubits.

  \item \textbf{High-fidelity operations:} Both single and two-qubit gates
    have demonstrated fidelities well above 99\%, among the highest in any
    quantum computing platform.

  \item \textbf{All-to-all connectivity:} In small ion chains, each ion can
    interact with any other ion in the chain, eliminating the need for swap
    networks found in architectures with nearest-neighbor connectivity.
\end{itemize}

\subsubsection*{Scaling Challenges and Solutions}

Despite their advantages, trapped ion systems face challenges in scaling to
larger numbers of qubits:

\begin{itemize}
  \item \textbf{Ion chain stability:} As the number of ions in a single chain
    increases, the chain becomes more susceptible to heating and instabilities.

  \item \textbf{Gate speed limitations:} The speed of entangling gates
    decreases as the ion chain grows, as motional modes become more densely
    packed in frequency space.

  \item \textbf{Laser control complexity:} Precisely addressing individual
    ions with lasers becomes more challenging in longer chains.
\end{itemize}

Proposed solutions to these scaling challenges include:

\begin{itemize}
  \item \textbf{Quantum charge-coupled device (QCCD) architecture:} Multiple
    small ion chains connected by ion transport between different trapping
    zones.

  \item \textbf{Photonic interconnects:} Using photons to entangle ions in
    physically separated traps.

  \item \textbf{Microfabricated trap arrays:} Creating integrated, scalable
    ion trap chips with multiple trapping regions and built-in control
    elements.

  \item \textbf{Sympathetic cooling:} Using auxiliary "coolant" ions of a
    different species to mitigate heating effects.
\end{itemize}

\subsubsection*{Leading Research Groups and Companies}

Major players in trapped ion quantum computing include:

\begin{itemize}
  \item \textbf{Academic groups:}
    \begin{itemize}
      \item University of Innsbruck (Rainer Blatt)
      \item University of Maryland/JQI (Christopher Monroe)
      \item NIST (David Wineland, Dietrich Leibfried)
      \item Oxford University (David Lucas)
    \end{itemize}

  \item \textbf{Companies:}
    \begin{itemize}
      \item IonQ (founded by Christopher Monroe and Jungsang Kim)
      \item Quantinuum (formerly Honeywell Quantum Solutions, merged with
        Cambridge Quantum Computing)
      \item Alpine Quantum Technologies (AQT)
      \item Universal Quantum
    \end{itemize}
\end{itemize}

\subsubsection*{Recent Achievements}

Recent milestones in trapped ion quantum computing include:

\begin{itemize}
  \item Demonstrations of quantum error correction codes
  \item Implementation of variational quantum algorithms and quantum simulations
  \item Realization of high-fidelity multi-qubit gates (3+ qubits)
  \item Creation of high-fidelity entangled states with 10+ qubits
  \item Demonstration of quantum advantage in certain specialized tasks
\end{itemize}

Under realistic noise models for trapped ion systems, fault-tolerant quantum
computation is achievable with gate error rates below approximately $10^{-4}$
using standard quantum error correction codes, making trapped ions a
promising platform for scalable quantum computation.

\dfn{Quantum Volume}{
  Quantum Volume is a hardware-agnostic metric introduced by IBM to
  characterize the capabilities of quantum computers. It measures the largest
  random circuit of equal width and depth that the quantum computer can
  successfully implement. Trapped ion quantum computers have demonstrated
  some of the highest quantum volumes among current platforms.
}

\nt{
  While superconducting qubits have received significant attention due to
  their rapid development by companies like IBM and Google, trapped ion
  systems continue to set records for gate fidelities and coherence times.
  The race between these technologies (and others like silicon spin qubits)
  remains ongoing, with each offering distinct advantages and challenges for
  practical quantum computing.
}

%%%%%%%%%%%%%%%%%%%%%%%%%%%%%%%%%%%%%%%%%%%%%%
% End of Lecture 22
%%%%%%%%%%%%%%%%%%%%%%%%%%%%%%%%%%%%%%%%%%%%%%

