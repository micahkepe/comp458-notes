\section{Lecture 20: Quantum Error Correction-- Bit Flip Errors}
\label{sec:lecture20}

Quantum circuits are highly sensitive to errors due to decoherence and
environmental noise, making error correction a cornerstone of reliable
quantum computation. This sensitivity stems from qubits losing their quantum
state through interactions with the environment, a process known as
decoherence. In quantum computing, significant effort is devoted not only to
compiler optimizations that reduce circuit depth but also to creating
redundancies within quantum circuits to protect information and enable
fault-tolerant computation.\footnote{For an overview, see
  \href{https://q-ctrl.com/topics/what-is-quantum-error-correction}{What is
quantum error correction?}} This lecture contrasts classical error correction
with quantum approaches, highlighting the unique challenges posed by quantum
mechanics.

\index{error correction!classical error detection}
\subsection*{Classical Error Correction}

Classical bits, which can be either 0 or 1, are prone to errors with a
probability \( p \in [0, 1] \), where the bit flips with probability \( p \)
and remains correct with probability \( 1 - p \). This is formally expressed as:

\[
  b =
  \begin{cases}
    \text{correct}, & \text{with probability } 1 - p \\
    \text{incorrect}, & \text{with probability } p
  \end{cases}
\]

Classical bits typically experience only bit-flip errors (e.g., 0 to 1 or 1
to 0), unlike quantum bits, which can also suffer phase flips due to their
superposition and entanglement properties.

\ex{Error Encoding of 3 for Classical Bit}{
  Suppose \( b = 0 \). Classically, we can build redundancies by making three
  copies of \( b \), sending \( 000 \), and taking a majority vote at the
  receiver to decode the original bit. The possible received states, their
  probabilities, and the decoded outcomes are shown in the table below:

  \begin{table}[H]
    \centering
    \begin{tabular}{lccc}
      \toprule
      Received State & Probability & Decoded Value & Correct? \\
      \midrule
      000 & \( (1 - p)^3 \) & 0 & Yes \\
      001 & \( (1 - p)^2 p \) & 0 & Yes \\
      010 & \( (1 - p)^2 p \) & 0 & Yes \\
      100 & \( (1 - p)^2 p \) & 0 & Yes \\
      011 & \( (1 - p) p^2 \) & 1 & No \\
      101 & \( (1 - p) p^2 \) & 1 & No \\
      110 & \( (1 - p) p^2 \) & 1 & No \\
      111 & \( p^3 \) & 1 & No \\
      \bottomrule
    \end{tabular}
    \caption{Probabilities of received states for a 3-bit repetition code.}
  \end{table}

  \vspace{0.3cm}

  The probability of correctly decoding \( b = 0 \) is the sum of
  probabilities where the majority is 0: \( (1 - p)^3 + 3 (1 - p)^2 p = 1 -
  3p^2 + 2p^3 \). This improves the success probability from the initial \( 1
  - p \) to \( 1 - 3p^2 + 2p^3 \).

  \vspace{0.3cm}

  For \( p = 0.01 \), the initial success rate without encoding is 0.99, but
  with three-bit redundancy, it increases to 0.999702, significantly
  enhancing reliability.
}

\vspace{0.3cm}

Interestingly, modern devices like smart phones employ similar redundancy
techniques, often using 5-7 redundancy bits in flash memory for error
correction. This contributes to power consumption due to the additional
storage and processing required, a trade-off for improved data integrity in
noisy environments.

\vspace{0.3cm}

\index{error correction!quantum error detection}
\subsection*{Qubit Flip Error Correction}
Quantum error correction is fundamentally more complex than its classical
counterpart due to the no-cloning theorem, which states that an unknown
quantum state cannot be perfectly copied.\footnote{For details on the
no-cloning theorem, see \href{https://learning.quantum.ibm.com/course/basics-of-quantum-information/quantum-circuits}{Quantum circuits}.}
This prevents us from simply making trivial copies of a qubit’s state, as in
classical systems. Even attempting to "clone" a qubit would not preserve the
original superposition \( |\psi\rangle = \alpha |0\rangle + \beta |1\rangle
\), rendering direct repetition impractical. Instead, quantum error
correction relies on encoding logical qubits into entangled states across
multiple physical qubits, using codes like the repetition code.

\vspace{0.3cm}

\index{error correction!repetition code}
\dfn{Repetition Code}{
  The quantum \textbf{repetition code} encodes a single logical qubit into
  multiple physical qubits to protect against errors. For a code distance of
  3, the state \( |\psi\rangle = \alpha |0\rangle + \beta |1\rangle \) is
  encoded into \( |\psi_3\rangle = \alpha |000\rangle + \beta |111\rangle \).
  This is achieved by starting with the initial qubit in \( |\psi\rangle \)
  and two ancilla qubits in \( |0\rangle \), then applying CNOT gates from
  the original qubit to the ancillas:

  \[
    \begin{quantikz}
      \lstick{$q_0 \: \ket{\psi}$} & \ctrl{1} & \ctrl{2} & \qw \\
      \lstick{$q_1 : \zero$} & \targ{} & \qw & \qw \\
      \lstick{$q_2 : \zero$} & \qw & \targ{} & \qw \\
    \end{quantikz}
  \]

  This circuit entangles the qubits such that a single bit-flip error (e.g.,
  \( X \) on one qubit) can be detected and corrected using syndrome
  measurements without collapsing the superposition.\footnote{See
    \href{https://www.nature.com/articles/s41567-023-02282-2}{Protecting
    expressive circuits with a quantum error detection code} for encoding
  techniques.}
}

\index{error correction!syndrome}
\dfn{Syndrome}{
  In quantum computing, \textbf{syndrome} measurement is a process used in
  quantum error correction (QEC) to identify errors without revealing the
  quantum information stored in the qubits. The syndrome measurement provides
  information about the type of error that has occurred, such as a bit flip
  or a phase flip, but it does not reveal the information stored in the
  logical qubit, thus preserving quantum superposition
}

\vspace{0.3cm}

To correct a single qubit flip error, consider the encoded state \(
|\psi_3\rangle = \alpha \ket{000} + \beta \ket{111} \). If an \( X \)
error occurs on \( q_1 \), the state becomes \( \alpha \ket{010} + \beta
\ket{101} \). Syndrome measurements are performed using two ancilla qubits
to measure the parities \( q_0 q_1 \) and \( q_1 q_2 \):

\begin{itemize}
  \item Measure \( X \otimes X \) between \( q_0 \) and \( q_1 \) with an
    ancilla initialized in \( |+\rangle \), applying CNOTs from \( q_0 \) and
    \( q_1 \) to the ancilla, then measuring it in the \( Z \)-basis. Outcome
    -1 indicates a flip on \( q_0 \) or \( q_1 \).
  \item Measure \( X \otimes X \) between \( q_1 \) and \( q_2 \) similarly.
    Outcome -1 indicates a flip on \( q_1 \) or \( q_2 \).
\end{itemize}

\vspace{0.3cm}

\noindent
For the error on \( q_1 \):
\begin{itemize}
  \item \( q_0 q_1 \): -1 (flip detected).
  \item \( q_1 q_2 \): -1 (flip detected).
  \item Syndrome (-1, -1) uniquely identifies \( q_1 \), and an \( X \) gate
    corrects it, restoring \( |\psi_3\rangle \).
\end{itemize}

This process, detailed in \href{https://www.ibm.com/quantum/blog/error-correction-codes}
{Error correcting codes for near-term quantum computers}, ensures correction
without measuring the logical state directly, preserving quantum information.

\aside{
  \textbf{Silent Data Corruption (SDC)} occurs when errors go undetected,
  posing significant risks in computing systems. In classical systems, SDC
  can corrupt data silently, leading to integrity issues in storage or
  processing. In quantum computing, undetected errors can propagate through
  entanglements, undermining computation reliability and producing incorrect
  results. This is particularly challenging due to the fragility of quantum
  states, making robust error correction essential to mitigate SDC and
  achieve fault tolerance.\footnote{See
    \href{https://physicsworld.com/a/why-error-correction-is-quantum-computings-defining-challenge}{Why
    error correction is quantum computing's defining challenge} for SDC
  implications.}
}

%%%%%%%%%%%%%%%%%%%%%%%%%%%%%%%%%%%%%%%%%%%%%%
% End of Lecture 20
%%%%%%%%%%%%%%%%%%%%%%%%%%%%%%%%%%%%%%%%%%%%%%

